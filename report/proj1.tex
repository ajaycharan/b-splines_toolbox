%%%%%%%%%%%%%%%%%%%%%%%%%%%%%%%%%%%%%%%%%
% Short Sectioned Assignment
% LaTeX Template
% Version 1.0 (5/5/12)
%
% This template has been downloaded from:
% http://www.LaTeXTemplates.com
%
% Original author:
% Frits Wenneker (http://www.howtotex.com)
%
% License:
% CC BY-NC-SA 3.0 (http://creativecommons.org/licenses/by-nc-sa/3.0/)
%
%%%%%%%%%%%%%%%%%%%%%%%%%%%%%%%%%%%%%%%%%

%----------------------------------------------------------------------------------------
%	PACKAGES AND OTHER DOCUMENT CONFIGURATIONS
%----------------------------------------------------------------------------------------

\documentclass[paper=a4, fontsize=11pt]{scrartcl} % A4 paper and 11pt font size

\usepackage[T1]{fontenc} % Use 8-bit encoding that has 256 glyphs
\usepackage{fourier} % Use the Adobe Utopia font for the document - comment this line to return to the LaTeX default
\usepackage[english]{babel} % English language/hyphenation
\usepackage{amsmath,amsfonts,amsthm} % Math packages
\usepackage{multicol}
\usepackage[margin=1in]{geometry}

\usepackage{sectsty} % Allows customizing section commands
\allsectionsfont{\centering \normalfont\scshape} % Make all sections centered, the default font and small caps
% \allsectionsfont{\normalfont\scshape} % Make all sections centered, the default font and small caps

\usepackage{fancyhdr} % Custom headers and footers
\pagestyle{fancyplain} % Makes all pages in the document conform to the custom headers and footers
\fancyhead{} % No page header - if you want one, create it in the same way as the footers below
\fancyfoot[L]{} % Empty left footer
\fancyfoot[C]{} % Empty center footer
\fancyfoot[R]{\thepage} % Page numbering for right footer
\renewcommand{\headrulewidth}{0pt} % Remove header underlines
\renewcommand{\footrulewidth}{0pt} % Remove footer underlines
\setlength{\headheight}{13.6pt} % Customize the height of the header

% \numberwithin{equation}{section} % Number equations within sections (i.e. 1.1, 1.2, 2.1, 2.2 instead of 1, 2, 3, 4)
% \numberwithin{figure}{section} % Number figures within sections (i.e. 1.1, 1.2, 2.1, 2.2 instead of 1, 2, 3, 4)
% \numberwithin{table}{section} % Number tables within sections (i.e. 1.1, 1.2, 2.1, 2.2 instead of 1, 2, 3, 4)

\setlength\parindent{0pt} % Removes all indentation from paragraphs - comment this line for an assignment with lots of text

%----------------------------------------------------------------------------------------
%	TITLE SECTION
%----------------------------------------------------------------------------------------

\newcommand{\horrule}[1]{\rule{\linewidth}{#1}} % Create horizontal rule command with 1 argument of height

\title{	
    \normalfont \normalsize 
    \textsc{Fundamentals of Linear Algebra and Optimization} \\ [25pt] % Your university, school and/or department name(s)
    \horrule{0.5pt} \\[0.4cm] % Thin top horizontal rule
    \huge Project 1 \\ % The assignment title
    \horrule{2pt} \\[0.5cm] % Thick bottom horizontal rule
}

\author{Armon Shariati} % Your name

\date{\normalsize\today} % Today's date or a custom date

\begin{document}

\maketitle % Print the title

%----------------------------------------------------------------------------------------
%	PROBLEM 1
%----------------------------------------------------------------------------------------

\section*{Problem P1}
\setcounter{section}{1}
\setcounter{subsection}{1}

\subsection{}

Given the linear system which can yield $d_{1}, ... , d_{N-1}$ in terms of
$x_{0}, ... , x_{N}$,

\begin{align*}
    \begin{bmatrix}
        \frac{7}{2} & 1 \\
        1 & 4 & 1 & & 0\\
          & \ddots & \ddots & \ddots \\
        0 & & 1 & 4 & 1\\
          & &   & 1 & \frac{7}{2}\\
    \end{bmatrix} \begin{bmatrix}
        d_{1}\\
        d_{2}\\
        \vdots\\
        d_{N-2}\\
        d_{N-1}\\
    \end{bmatrix} = \begin{bmatrix}
        6x_{1} - \frac{3}{2}d_{0}\\
        6x_{2}\\
        \vdots\\
        6x_{N-2}\\
        6x_{N-1} - \frac{3}{2}d_{N}\\
    \end{bmatrix}
\end{align*}

When evaluated and simplified, yields the following equations,

\begin{align*}
    \begin{split}
        \frac{1}{4}d_{0} + \frac{7}{12}d_{1} + \frac{1}{6}d_{2} &= x_{1}\\
        \frac{1}{6}d_{1} + \frac{4}{6}d_{2} + \frac{1}{6}d_{3} &= x_{2}\\
        \vdots\\
        \frac{1}{6}d_{N-3} + \frac{4}{6}d_{N-2} + \frac{1}{6}d_{N-1} &= x_{N-2}\\
        \frac{1}{6}d_{N-2} + \frac{7}{12}d_{N-1} + \frac{1}{4}d_{N} &= x_{N-1}.\\
    \end{split}
\end{align*}

These $x$'s correspond to every $1^{st}$ and $4^{th}$ control point of all
cubic bezier curve segments derived from the de Boor control points, e.g.
$x_{1}$ and $x_{2}$ correspond to $b_{0}^{2}$ and $b_{3}^{2}$, $x_{2}$ and
$x_{3}$ correspond to $b_{0}^{3}$ and $b_{3}^{3}$, and $x_{i}$ and $x_{i+1}$
correspond to $b_{0}^{i+1}$ and $b_{3}^{i+1}$, etc. 

\bigskip
Furthermore, these points $x_{1}, ..., x_{n}$ can be used to compute $d_{1},
... , d_{N-1}$, because over the course of deCasteljau's algorithm, these
points remain constant. All of the other control points in between become
forgotten with each subdivision. As a result, these points are guaranteed to
remain on the polygonal line approximating the curve regardless of the depth
of recursion. 


\end{document}
