%%%%%%%%%%%%%%%%%%%%%%%%%%%%%%%%%%%%%%%%%
% Short Sectioned Assignment
% LaTeX Template
% Version 1.0 (5/5/12)
%
% This template has been downloaded from:
% http://www.LaTeXTemplates.com
%
% Original author:
% Frits Wenneker (http://www.howtotex.com)
%
% License:
% CC BY-NC-SA 3.0 (http://creativecommons.org/licenses/by-nc-sa/3.0/)
%
%%%%%%%%%%%%%%%%%%%%%%%%%%%%%%%%%%%%%%%%%

%----------------------------------------------------------------------------------------
%	PACKAGES AND OTHER DOCUMENT CONFIGURATIONS
%----------------------------------------------------------------------------------------

\documentclass[paper=a4, fontsize=11pt]{scrartcl} % A4 paper and 11pt font size

\usepackage[T1]{fontenc} % Use 8-bit encoding that has 256 glyphs
\usepackage{fourier} % Use the Adobe Utopia font for the document - comment this line to return to the LaTeX default
\usepackage[english]{babel} % English language/hyphenation
\usepackage{amsmath,amsfonts,amsthm} % Math packages
\usepackage{multicol}
\usepackage[margin=1in]{geometry}

\usepackage{sectsty} % Allows customizing section commands
\allsectionsfont{\centering \normalfont\scshape} % Make all sections centered, the default font and small caps
% \allsectionsfont{\normalfont\scshape} % Make all sections centered, the default font and small caps

\usepackage{fancyhdr} % Custom headers and footers
\pagestyle{fancyplain} % Makes all pages in the document conform to the custom headers and footers
\fancyhead{} % No page header - if you want one, create it in the same way as the footers below
\fancyfoot[L]{} % Empty left footer
\fancyfoot[C]{} % Empty center footer
\fancyfoot[R]{\thepage} % Page numbering for right footer
\renewcommand{\headrulewidth}{0pt} % Remove header underlines
\renewcommand{\footrulewidth}{0pt} % Remove footer underlines
\setlength{\headheight}{13.6pt} % Customize the height of the header

% \numberwithin{equation}{section} % Number equations within sections (i.e. 1.1, 1.2, 2.1, 2.2 instead of 1, 2, 3, 4)
% \numberwithin{figure}{section} % Number figures within sections (i.e. 1.1, 1.2, 2.1, 2.2 instead of 1, 2, 3, 4)
% \numberwithin{table}{section} % Number tables within sections (i.e. 1.1, 1.2, 2.1, 2.2 instead of 1, 2, 3, 4)

\setlength\parindent{0pt} % Removes all indentation from paragraphs - comment this line for an assignment with lots of text

%----------------------------------------------------------------------------------------
%	TITLE SECTION
%----------------------------------------------------------------------------------------

\newcommand{\horrule}[1]{\rule{\linewidth}{#1}} % Create horizontal rule command with 1 argument of height

\title{	
    \normalfont \normalsize 
    \textsc{Fundamentals of Linear Algebra and Optimization} \\ [25pt] % Your university, school and/or department name(s)
    \horrule{0.5pt} \\[0.4cm] % Thin top horizontal rule
    \huge Project 3 \\ % The assignment title
    \horrule{2pt} \\[0.5cm] % Thick bottom horizontal rule
}

\author{Danica Fine and Armon Shariati} % Your name

\date{\normalsize\today} % Today's date or a custom date

\begin{document}

\maketitle % Print the title


\section*{Problem 0}
\setcounter{section}{0}
\setcounter{subsection}{1}

We wish to prove that the tangent vectors $m_0$ at $x_0$ and $m_N$ at $x_N$ are given by
\begin{align*}
    m_0 &= 3(d_0 - x_0) \\
    m_N &= 3(x_N - d_N).
\end{align*}
Since the curves are given by $C(t)$ and the set of control points, $b_0,
\dots, b_3$, we compute the derivative of $C(t)$ and evaluate at $0$ and $1$ to
get $m_0$ and $m_n$, respectively.
\begin{align*}
    C(t) &=
    (1-t)^3 b_0 + 3(1-t)^2 t b_1 + 3(1-t)t^2 b_2 + t^3 b_3 \\
    C^{\prime}(t) &=
    -3(1-t)^2b_0 - 6(1-t)tb_1 + 3(1-t)^2b_1 -3t^2b_2 + 6(1-t)tb_2 +     3t^2b_3 \\
    &=
    3(1-t)^2(b_1 - b_0) + 6t(1-t)(b_2 - b_1) + 3t^2(b_3 - b_2)
\end{align*}
So for $m_0$, we get
\begin{align*}
    C^{\prime}(0) &=
    3(1)^2(b_1 - b_0) + 6(0)(1-t)(b_2 - b_1) + 3(0)^2(b_3 - b_2) \\
    &=
    3(b_1 - b_0) \\
    m_0 &= 3(d_0 - x_0),
\end{align*}
where the final substitution is made from equations given to us in the project
instructions for the first curve. Similarly, for $m_N$ we acquire
\begin{align*}
    C^{\prime}(1) &=
    3(0)^2(b_1 - b_0) + 6(1)(0)(b_2 - b_1) + 3(1)^2(b_3 - b_2) \\
    &=
    3(b_3 - b_2) \\
    m_0 &= 3(x_N - d_N),
\end{align*}
where the substitutions are provided to us for the final curve, $C_N$.


\section*{Problem 1}
\setcounter{section}{1}
\setcounter{subsection}{1}

From $C^\prime(t)$ above, we compute $C^{\prime\prime}(t)$
\begin{align*}
    C^{\prime\prime}(t) &=
    -6(1-t)^2(b_1 - b_0) + 6((1-t) - t)(b_2 - b_1) + 6t(b_3 - b_2) \\
    &=
    - 6(1-t)b_1 + 6(1-t)b_0 + 6(1-t)(b_2 - b_1) - 6t(b_2 - b_1) +         6tb_3 - 6tb_2.
\end{align*}
For $C_1$, we evaluate $C^{\prime\prime}(t)$ for $t=0$ to get
\begin{align*}
    C^{\prime\prime}(0) &=
    - 6(0)b_1 + 6(0)b_0 + 6(0)(b_2 - b_1) - 6(1)(b_2 - b_1) +             6(1)b_3 - 6(1)b_2 \\
    &=
    -6b_2 + 6 b_1 + 6b_3 - 6b_2 \\
    &=
    6(b_1 - 2b_2 + b_3),
\end{align*}
and for $C_N$, we evaluate the expression for $t=1$, which gives us
\begin{align*}
    C^{\prime\prime}(1) &=
    - 6(1)b_1 + 6(1)b_0 + 6(1)(b_2 - b_1) - 6(0)(b_2 - b_1) +             6(0)b_3 - 6(0)b_2 \\
    &=
    -6b_1 + 6 b_0 + 6b_2 - 6b_1 \\
    &=
    6(b_0 - 2b_1 + b_2).
\end{align*}

Given the linear system

\begin{align*}
    \begin{pmatrix}
        \frac{7}{2} & 1 \\
        1 & 4 & 1 & & 0\\
          & \ddots & \ddots & \ddots \\
        0 & & 1 & 4 & 1\\
          & &   & 1 & \frac{7}{2}\\
    \end{pmatrix} \begin{pmatrix}
        d_{1}\\
        d_{2}\\
        \vdots\\
        d_{N-2}\\
        d_{N-1}\\
    \end{pmatrix} = \begin{pmatrix}
        6x_{1} - \frac{3}{2}d_{0}\\
        6x_{2}\\
        \vdots\\
        6x_{N-2}\\
        6x_{N-1} - \frac{3}{2}d_{N}\\
    \end{pmatrix}
\end{align*}

and 

\begin{align*}
    d_{0} &= \frac{2}{3} x_{0} + \frac{1}{3} d_{1}\\
    d_{N} &= \frac{1}{3} d_{N-1} + \frac{2}{3} x_{N},\\
\end{align*}

we can substitute $d_{0}$ and $d_{N}$ into the following equations

\begin{align*}
    \frac{7}{2}d_{1} + d_{2} &= 6x_{1} - \frac{3}{2}d_{0}\\
    d_{N-2} + \frac{7}{2}d_{N-1}  &= 6x_{N-1} - \frac{3}{2}d_{N}.\\
\end{align*}

As a result,

\begin{align*}
    \begin{split}
        \frac{7}{2}d_{1} + d_{2} &= 6x_{1} - \frac{3}{2}\left( \frac{2}{3} x_{0} + \frac{1}{3} d_{1} \right)\\
        \frac{7}{2}d_{1} + d_{2} &= 6x_{1} - x_{0} - \frac{1}{2} d_{1}\\
        4d_{1} + d_{2} &= 6x_{1} - x_{0}\\
    \end{split}
    \begin{split}
        d_{N-2} + \frac{7}{2}d_{N-1} &= 6x_{1} - \frac{3}{2}\left( \frac{1}{3} d_{N-1} + \frac{2}{3} x_{N} \right)\\
        d_{N-2} + \frac{7}{2}d_{N-1} &= 6x_{1} - \frac{1}{2} d_{N-1} - x_{N}\\
        d_{N-2} + 4d_{N-1} &= 6x_{1} - x_{N}.\\
    \end{split}
\end{align*}

Therefore, the original system becomes

\begin{align*}
    \begin{pmatrix}
        4 & 1 \\
        1 & 4 & 1 & & 0\\
          & \ddots & \ddots & \ddots \\
        0 & & 1 & 4 & 1\\
          & &   & 1 & 4\\
    \end{pmatrix} \begin{pmatrix}
        d_{1}\\
        d_{2}\\
        \vdots\\
        d_{N-2}\\
        d_{N-1}\\
    \end{pmatrix} = \begin{pmatrix}
        6x_{1} - x_{0}\\
        6x_{2}\\
        \vdots\\
        6x_{N-2}\\
        6x_{N-1} - x_{N}\\
    \end{pmatrix}
\end{align*}

\section*{Problem 2}
\setcounter{section}{2}
\setcounter{subsection}{1}

Given the linear system

\begin{align*}
    \begin{pmatrix}
        \frac{7}{2} & 1 \\
        1 & 4 & 1 & & 0\\
          & \ddots & \ddots & \ddots \\
        0 & & 1 & 4 & 1\\
          & &   & 1 & \frac{7}{2}\\
    \end{pmatrix} \begin{pmatrix}
        d_{1}\\
        d_{2}\\
        \vdots\\
        d_{N-2}\\
        d_{N-1}\\
    \end{pmatrix} = \begin{pmatrix}
        6x_{1} - \frac{3}{2}d_{0}\\
        6x_{2}\\
        \vdots\\
        6x_{N-2}\\
        6x_{N-1} - \frac{3}{2}d_{N}\\
    \end{pmatrix}
\end{align*}

and 

\begin{align*}
    d_{0} &= d_{1} + \frac{2}{3} x_{0} - \frac{2}{3} x_{1}\\
    d_{N} &= d_{N-1} + \frac{2}{3} x_{N} - \frac{2}{3} x_{N-1}\\
\end{align*}

we can substitute $d_{0}$ and $d_{N}$ into the following equations

\begin{align*}
    \frac{7}{2}d_{1} + d_{2} &= 6x_{1} - \frac{3}{2}d_{0}\\
    d_{N-2} + \frac{7}{2}d_{N-1}  &= 6x_{N-1} - \frac{3}{2}d_{N}.\\
\end{align*}

As a result,

\begin{align*}
    \begin{split}
        \frac{7}{2}d_{1} + d_{2} &= 6x_{1} - \frac{3}{2}\left( d_{1} + \frac{2}{3} x_{0} - \frac{2}{3} x_{1} \right)\\
        \frac{7}{2}d_{1} + d_{2} &= 6x_{1} - \frac{3}{2} d_{1} - x_{0} + x_{1}\\
        5d_{1} + d_{2} &= 7x_{1} - x_{0}\\
    \end{split}
    \begin{split}
        d_{N-2} + \frac{7}{2}d_{N-1} &= 6x_{1} - \frac{3}{2}\left( d_{N-1} + \frac{2}{3} x_{N} - \frac{2}{3} x_{N-1} \right)\\
        d_{N-2} + \frac{7}{2}d_{N-1} &= 6x_{1} - \frac{3}{2} d_{N-1} - x_{N} + x_{N-1}\\
        d_{N-2} + 5d_{N-1} &= 7x_{N-1} - x_{N}\\
    \end{split}
\end{align*}

Therefore, the original system becomes

\begin{align*}
    \begin{pmatrix}
        5 & 1 \\
        1 & 4 & 1 & & 0\\
          & \ddots & \ddots & \ddots \\
        0 & & 1 & 4 & 1\\
          & &   & 1 & 5\\
    \end{pmatrix} \begin{pmatrix}
        d_{1}\\
        d_{2}\\
        \vdots\\
        d_{N-2}\\
        d_{N-1}\\
    \end{pmatrix} = \begin{pmatrix}
        7x_{1} - x_{0}\\
        6x_{2}\\
        \vdots\\
        6x_{N-2}\\
        7x_{N-1} - x_{N}\\
    \end{pmatrix}
\end{align*}

\section*{Problem 3}
We want to show that
\begin{equation}
    m_0 = C^\prime (0) = b_1 - x_0 = - \frac{3}{2}x_0 +2x_1 - \frac{1}{2}. \notag
\end{equation}
By definition, $m_i$ is the vector tangent to the point $x_i$. We know, then,
that $m_0 = C^\prime(0)$. Thus, we may proceed as follows:
\begin{align*}
    C(t) &= \frac{(2 - t)^2}{4} x_0 + \frac{(2-t)t}{2} b_1 + \frac{t^2}{4}x_2 \\
    C^\prime (t) &= \frac{(2t - 4)}{4}x_0 - \frac{t}{2}b_1 + \frac{(2-t)}{2}b_1 + \frac{t}{2}x_2 \\
    C^\prime(0) &= \frac{(2(0) - 4)}{4}x_0 - \frac{(0)}{2}b_1 + \frac{(2-(0))}{2}b_1 + \frac{(0)}{2}x_2 \\
                &= -x_0 -0 + b_1 +0 \\
                &= b_1 - x_0.
\end{align*}
From a previous result, we know that $b_1 = -\frac{1}{2}x_0 + 2x_1
-\frac{1}{2}x_2$. Substituting this expression in for $b_1$ of our computed
value of $C^\prime(0)$, we acquire
\begin{align*}
    m_0 &= b_1 - x_0 \\
        &= (-\frac{1}{2}x_0 + 2x_1 -\frac{1}{2}x_2) - x_0 \\
        &= -\frac{3}{2}x_0 + 2x_1 -\frac{1}{2}x_2 ,
\end{align*}
as desired.

Given that $d_0 = x_0 + \frac{1}{3}m_0$, we can expand using the above result,
\begin{align*}
    d_0 &= x_0 + \frac{1}{3}m_0 \\
        &= x_0 + \frac{1}{3}(-\frac{3}{2}x_0 + 2x_1 -\frac{1}{2}x_2) \\
        &= x_0 - \frac{1}{2}x_0 + \frac{2}{3}x_1 - \frac{1}{6}x_2 \\
        &= \frac{1}{2}x_0 + \frac{1}{2}(\frac{4}{3}x_1 - \frac{1}{3}x_2) \\
        &= \frac{1}{2}x_0 + \frac{1}{2}(x_1 + \frac{1}{3}x_1 + \frac{1}{3}x_2) \\
        &= \frac{1}{2}x_0 + \frac{1}{2}(x_1 + \frac{1}{3}(x_1 - x_2))
\end{align*}
and we arrive at the desired expression.

Now we wish to show that
\begin{equation}
    m_N = C^\prime (2) = x_N - b_N = \frac{1}{2}x_{N-2} - 2x_{N-1} + \frac{3}{2}x_N. \notag
\end{equation}
Similar to the above, we proceed as follows:
\begin{align*}
    C(t) &= \frac{(2 - t)^2}{4} x_{N-1} + \frac{(2-t)t}{2} b_N + \frac{t^2}{4}x_N \\
    C^\prime (t) &= \frac{(2t - 4)}{4}x_{N-1} - \frac{t}{2}b_N + \frac{(2-t)}{2}b_N + \frac{t}{2}x_N \\
    C^\prime(2) &= \frac{(2(2) - 4)}{4}x_{N-1} - \frac{(2)}{2}b_N + \frac{(2-(2))}{2}b_N + \frac{(2)}{2}x_N \\
                &= \frac{0}{4}x_{N-1} - \frac{2}{2}b_N + \frac{0}{2}b_N + \frac{2}{2}x_N \\
                &= 0 -b_1 + 0 + x_N \\
                &= x_N - b_N.
\end{align*}
We are given that $b_N = -\frac{1}{2}x_{N-2} + 2x_{N-1} -\frac{1}{2}x_N$.
Substituting this expression in for $b_1$ of our computed value of
$C^\prime(2)$, we acquire
\begin{align*}
    m_N &= x_N - b_N \\
        &= x_N - (-\frac{1}{2}x_{N-2} + 2x_{N-1} -\frac{1}{2}x_N) \\
        &= \frac{1}{2}x_{N-2} - 2x_{N-1} +\frac{3}{2}x_N ,
\end{align*}
as desired.

Given that $d_N = x_N - \frac{1}{3}m_N$, we can expand using the above result,
\begin{align*}
    d_N &= x_N - \frac{1}{3}m_N \\
        &= x_N - \frac{1}{3}(\frac{1}{2}x_{N-2} - 2x_{N-1} +\frac{3}{2}x_N) \\
        &= x_N - \frac{1}{6}x_{N-2} + \frac{2}{3}x_{N-1} - \frac{1}{2}x_N \\
        &= \frac{1}{2}(\frac{4}{3}x_{N-1} - \frac{1}{3}x_{N-2})+ \frac{1}{2}x_N\\
        &= \frac{1}{2}(x_{N-1} + \frac{1}{3}(x_{N-1}-x_{N-2}))+ \frac{1}{2}x_N,
\end{align*}
and we arrive at the desired expression.

\section*{Problem 4}
First, we want to show that the third derivation at $b_0$ and $b_3$ of a Bezier
cubic specified by the control points $(b_0, b_1, b_2, b_3)$ is $6(-b_0 + 3b_1
-3b_1 + b_3)$. We begin with the function $C(t)$.
\begin{align*}
    C(t) &=
    (1 - t)^3b_0 + 3(1-t)^2b_1 +3(1-t)t^2b_2 + t^3b_3 \\
    C^\prime (t) &=
    3(1-t)^2 (-1) b_0 + 3 \lbrack 2(1-t)(-1)t + (1-t)^2 \rbrack b_1 + 3     \lbrack (-1)t^2 + (2)(1-t)t \rbrack b_2 + 3t^2 b_3 \\
    &=
    -3(1-t)^2b_0 -6(1-t)t b_1 + 3(1-t)^2b_1 -3t^2b_2 + 6(1-t)t b_2             +3t^2b_3 \\
    C^{\prime \prime}(t) &=
    3(2)(1-t)b_0 -6 \lbrack (-1)t + (1-t)\rbrack b_1 - 3(2)(1-t)b_1 -         3(2)t b_2 + 6 \lbrack -t + (1-t)\rbrack b_2 + 6t b_3 \\
    &=
    6(1-t)b_0 + 6t b_1 -6(1-t) b_1 -6(1-t) b_1-6t b_2 -6t b_2 +6(1-t)         b_2 + 6t b_3 \\
    C^{\prime  \prime \prime}(t) &=
    6(-1)b_0 + 6b_1 - 6(-1)b_1 -6(-1)b_1 - 6b_2 - 6 b_2 + 6(-1)b_2 +         6b_2 \\
    &=
    6(-b_0 + b_1 + b_1 + b_1 - b_2 - b_2 - b_2 + b_3) \\
    &=
    6(-b_0 +3 b_1 - 3b_2 + b_3)
\end{align*}
Now, we wish to show that when $N=3$, there is no need to solve a linear
system, and that the points $d_0, d_1, d_2, d_3$ are given in terms of $x_0,
x_1, x_2, x_3$.

It is given that $C^{\prime\prime\prime}_1(1) = C^{\prime\prime\prime}_2(0)$. So we have
\begin{align*}
    6(-b_0^1 + 3b_1^1 -3b_2^1 +b_3^1) &=
    6(-b_0^2 + 3b_1^2 -3b_2^2 +b_3^2) \\
    -b_0^1 + 3b_1^1 -3b_2^1 +b_3^1 &=
    -b_0^2 + 3b_1^2 -3b_2^2 +b_3^2 .
\end{align*}
Using the set of equations from page $3$ of the project instructions, we can substitute in for the $b_i^j$ and solve for $d_0$.
\begin{align*}
    -(x_0) + 3(d_0) - 3(\frac{1}{2}d_0 + \frac{1}{2}d_1) + x_1 &=
    -(x_1) + 3(\frac{2}{3} d_1 + \frac{1}{3}d_2) - 3(\frac{1}{3}d_1 +         \frac{2}{3}d_2) + x_2 \\
    -x_0 + \frac{3}{2}d_0 - \frac{3}{2}d_1 + x_1 &=
    -x_1 + d_1 -d_2 + x_2 \\
    \frac{3}{2}d_0 &=
    x_0 - 2x_1 +x_2 + \frac{3}{2}d_1 + d_1 - d_2 \\
    d_0 &=
    \frac{2}{3}x_0 - \frac{4}{3}x_1 + \frac{2}{3}x_2 + \frac{5}{3}d_1 -     \frac{2}{3}d_2
\end{align*}
From the original linear system, we obtain
\begin{align*}
    \frac{7}{2}d_1 + d_2 &=
    6x_1 - \frac{3}{2}d_0 \\
    \frac{7}{2}d_1 &=
    6x_1 - \frac{3}{2}d_0 - d_2 \\
    d_1 &=
    (\frac{2}{7})6x_1 -(\frac{2}{7}) \frac{3}{2}d_0 -(\frac{2}{7})d_2 \\
    &=
    \frac{12}{7}x_1 - \frac{3}{7}d_0 - \frac{2}{7} d_2 \\
    \frac{5}{3}d_1 &=
    (\frac{5}{3})\frac{12}{7}x_1 - (\frac{5}{3})\frac{3}{7}d_0 -             (\frac{5}{3})\frac{2}{7}d_2 \\
    \frac{5}{3}d_1 &=
    \frac{20}{7}x_1 - \frac{5}{7}d_0 - \frac{10}{21}d_2
\end{align*}
and can substitute this into the above equation for $d_0$. This yields
\begin{align*}
    d_0 &=
    \frac{2}{3}x_0 - \frac{4}{3}x_1 + \frac{2}{3}x_2 + (\frac{20}{7}x_1     - \frac{5}{7}d_0 - \frac{10}{21}d_2) - \frac{2}{3}d_2 \\
    \frac{12}{7}d_0 &=
    \frac{2}{3}x_0 - \frac{32}{21}x_1 +\frac{2}{3}x_2 -\frac{8}{7}d_2 \\
    d_0 &=
    (\frac{7}{12})\frac{2}{3}x_0 - (\frac{7}{12})\frac{32}{21}x_1 +         (\frac{7}{12})\frac{2}{3}x_2 - (\frac{7}{12})\frac{8}{7}d_2 \\
    &=
    \frac{7}{18}x_0 + \frac{8}{9}x_1 + \frac{7}{18}x_2 - \frac{2}{3}d_2,
\end{align*}
which is the desired equation of $d_0$. This equation can then be substituted into the one we acquired from the original linear system
\begin{align*}
    d_1 &=
    \frac{12}{7}x_1 - \frac{3}{7}d_0 - \frac{2}{7} d_2 \\
    &=
    \frac{12}{7}x_1 - \frac{3}{7}(\frac{7}{18}x_0 + \frac{8}{9}x_1 +         \frac{7}{18}x_2 - \frac{2}{3}d_2) - \frac{2}{7} d_2 \\
    &=
    \frac{12}{7}x_1 - \frac{1}{6}x_0 - \frac{8}{21}x_1 - \frac{1}        {6}x_2     + \frac{2}{7}d_2 - \frac{2}{7} d_2 \\
    &=
    - \frac{1}{6}x_0 + \frac{4}{3}x_1 - \frac{1}{6}x_2
\end{align*}
which gives us the desired equation for $d_1$. One may note that the process
for acquiring $d_2$ and $d_3$ is symmetric (Prof. Gallier said that it was
alright to point this out and not show the work), and we arrive at the
following system:
\begin{align*}
    d_0 &=
    \frac{7}{18}x_0 +\frac{8}{9}x_1 +\frac{7}{18}x_2 -\frac{2}{3}d_2\\
    d_1 &=
    -\frac{1}{6}x_0 + \frac{4}{3}x_1 - \frac{1}{6}x_2 \\
    d_2 &=
    -\frac{1}{6}x_1 + \frac{4}{3}x_2 - \frac{1}{6}x_3 \\
    d_3 &=
    \frac{7}{18}x_1 +\frac{8}{9}x_2 +\frac{7}{18}x_3 -\frac{2}{3}d_1.
\end{align*}
Since $d_1$ and $d_2$ are computed entirely from the $x_i$, and $d_0$ and $d_3$
depend on the $x_i$, $d_1$, and $d_2$, it is obvious that we need not solve any
linear system when $N=3$.

Using a similar process, we seek to find the system of equations for when
$N=4$. To begin, we note that the equations which were used to find $d_0$ and
$d_N$ in the previous part of the problem are still valid for this curve
because the endpoints will be computed the same way. Similarly, the second and
second to last points are calculated in the same manner. Thus, we know that
\begin{align*}
    d_0 &=
    \frac{7}{18}x_0 +\frac{8}{9}x_1 +\frac{7}{18}x_2 -\frac{2}{3}d_2\\
    d_1 &=
    -\frac{1}{6}x_0 + \frac{4}{3}x_1 - \frac{1}{6}x_2 \\
    d_3 &=
    -\frac{1}{6}x_1 + \frac{4}{3}x_2 - \frac{1}{6}x_3 \\
    d_4 &=
    \frac{7}{18}x_1 +\frac{8}{9}x_2 +\frac{7}{18}x_3 -\frac{2}{3}d_2,
\end{align*}
and we have only to conclude that
\begin{align*}
    d_2 &=
    \frac{3}{2}x_2 - \frac{1}{4}d_1 - \frac{1}{4}d_3.
\end{align*}
To do so, we refer back to the linear system on page $4$ of the project instructions. From this, we can obtain
\begin{align*}
    d_1 + 4d_2 + d_3 &=
    6x_2 \\
    4d_2 &=
    6x_2 - d_1 - d_3 \\
    d_2 &=
    \frac{3}{2}x_2 - \frac{1}{4}d_1 - \frac{1}{4}d_3,
\end{align*}
as required.

When $N \geq 5$, the process for calculating $d_0, d_1, d_{N-1},$ and $d_N$ is the same as the above, so we have
\begin{align*}
    d_0 &=
    \frac{7}{18}x_0 +\frac{8}{9}x_1 +\frac{7}{18}x_2 -\frac{2}{3}d_2\\
    d_1 &=
    -\frac{1}{6}x_0 + \frac{4}{3}x_1 - \frac{1}{6}x_2 \\
    d_{N-1} &=
    -\frac{1}{6}x_{N-2} + \frac{4}{3}x_{N-1} - \frac{1}{6}x_N \\
    d_N &=
    \frac{7}{18}x_{N-2} +\frac{8}{9}x_{N-1} +\frac{7}{18}x_N -\frac{2}     {3}d_{N-2}.
\end{align*}
The remaining values $d_2, \dots, d_{N-2}$ are derived from the original linear
system given on page $3$ of the project instructions -- the same equation we
used to derive $d_2$ in the above part. Instead of deriving $d_2$ directly, we
reorganize the expression to obtain
\begin{align*}
    d_1 + 4d_2 + d_3 &=
    6x_2 \\
    4d_2 + d_3 &=
    6x_2 - d_1 \\
    &=
    6x_2 - (-\frac{1}{6}x_0 + \frac{4}{3}x_1 - \frac{1}{6}x_2) \\
    &=
    6x_2 + \frac{1}{6}x_0 - \frac{4}{3}x_1 + \frac{1}{6}x_2,
\end{align*}
which gives us the correct solution to the linear system. The process is
symmetric for $d_{N-2}$. The remaining entries are derived directly from the
original system, as we will obtain
\begin{align*}
    d_i + 4d_{i+1} + d_{i+2} = 6x_{i+1}
\end{align*}
for $2 \leq i \leq (N-2)$.

Because the system is represented by an $(N-3) \times (N-3)$ matrix, when
$N=5$, the given linear system is reduced to the $2 \times 2$ system formed by
the first and last columns of the original:
\begin{align*}
    \begin{pmatrix}
        4 & 1 \\
        1 & 4
    \end{pmatrix}
    \begin{pmatrix}
        d_2 \\
        d_3
    \end{pmatrix}
    =
    \begin{pmatrix}
        6x_2 + \frac{1}{6}x_0 - \frac{4}{3}x_1 + \frac{1}{6}x_2 \\
        6x_3 + \frac{1}{6}x_3 - \frac{4}{3}x_4 + \frac{1}{6}x_5
    \end{pmatrix}
\end{align*}

\section*{Problem 5}
\setcounter{section}{5}
\setcounter{subsection}{1}

Executing \texttt{project\_3.m} launches a graphical interface where the user
can click a series of points which will interpolate a b-spline curve. The first
curve is drawn after at least $5$ points are clicked and redrawn for every
subsequent point. With every point added, the tridiagonal system $A*d = x$
expands by $1$ and must be solved again. The program solves the system at
each iteration with a call to \texttt{solvetri.m}. This function solves 
the tridiagonal system using both Gaussian elimination and LU decomposition,
and outputs the time each process takes to the console.

\bigskip
It is indeed true that Gaussian elimination does not require any pivoting 
when solving interpolation problems.

\section*{Problem 6}
\setcounter{section}{6}
\setcounter{subsection}{1}

Executing \texttt{project\_3.m} launches a graphical interface where the user
can click a series of points which will interpolate a b-spline curve. The first
curve is drawn after at least $5$ points are clicked and redrawn for every
subsequent point. With every point added, the tridiagonal system $A*d = x$
expands by $1$ and must be solved again. The program solves the system at
each iteration with a call to \texttt{solvetri.m}. This function solves 
the tridiagonal system using both Gaussian elimination and LU decomposition,
and outputs the time each process takes to the console.

\bigskip
When the dimension of $A$ is small, the system is solved faster using Gaussian
elimination than using LU decomposition. However, as the system grows larger
the complexity of Gaussian elimination causes it to take much longer than LU
decomposition. The user can verify this by clicking more points to add to the
system and observing the increasing computation time of Gaussian elimination.

\bigskip
Using alternative tridiagonal systems with large diagonal entries, we can also
see that LU factorization is more numerically stable than Gaussian elimination.

\section*{Problem 7}
\setcounter{section}{7}
\setcounter{subsection}{1}

Executing \texttt{project\_3.m} launches a graphical interface where the user
can click a series of points which will interpolate a b-spline curve. The first
curve is drawn after at least $5$ points are clicked and redrawn for every
subsequent point. With every point added, the tridiagonal system $A*d = x$
expands by $1$ and must be solved again. The program solves the system at
each iteration with a call to \texttt{solvetri.m}. This function solves 
the tridiagonal system using both Gaussian elimination and LU decomposition,
and outputs the time each process takes to the console.

\bigskip
The main function \texttt{project\_3.m} calls the function
\texttt{curveinterp.m} to build and solve the tridiagonal system with various
end conditions. Modifying the \texttt{econd} parameter in
\texttt{project\_3.m}, changes which end condition \texttt{curveinterp.m} will
use. This parameter can be set to \texttt{natural}, \texttt{quadratic},
\texttt{bessel}, and \texttt{knot}. The user will see the effects of each
choice of the parameter as the curve is drawn.

\end{document}
